\documentclass{beamer}  


%\usepackage{beamerthemesplit}

%\usepackage{colortbl}
%\usepackage{algorithm, algorithmic}
%\usepackage{tikz}
\usepackage[BoldFont,SlantFont,CJKchecksingle]{xeCJK}
\usepackage{graphicx}
\usepackage{multicol}

\setCJKmainfont[BoldFont=SimHei,SlantedFont=KaiTi]{SimSun}
\setCJKsansfont[BoldFont=SimHei,SlantedFont=KaiTi]{SimSun}
\setCJKmonofont[ItalicFont={Adobe Fangsong Std}]{SimSun}
\setCJKfamilyfont{zhsong}{SimSun}
\setCJKfamilyfont{zhhei}{SimHei}
\setCJKfamilyfont{zhkai}{KaiTi}
\setCJKfamilyfont{zhfs}{FangSong}
\usetheme{CambridgeUS}  
\parindent 2em 

  
%\usepackage{fontspec}  

\title{人工神经网络}   
\author{肖强 \ 张哲槟\\密尚华}
\institute{}

\begin{document}  
  
\begin{frame}
\titlepage
\end{frame}

\begin{frame}
	\frametitle{内容}
	
	\begin{multicols}{2}
		\tableofcontents
	\end{multicols}
%\tableofcontents
\end{frame}

\section{人工神经网络简介}

\begin{frame} 
\frametitle{人工神经网络(Artificial Neural Network,ANN )}  
\begin{columns}

\column{.5\textwidth}
神经网络是一种运算模型,由大量的节点(或称神经元)之间相互联接构成。
%每个节点代表一种特定的输出函数,称为激励函数(activation function)。每两个节点间的连接都代表一个对于通过该连接信号的加权值,称之为权重,这相当于人工神经网络的记忆。网络的输出则依网络的连接方式,权重值和激励函数的不同而不同。
\\ 网络自身通常都是对自然界某种算法或者函数的逼近,也可能是对一种逻辑策略的表达。

\column{.5\textwidth}
	\begin{figure}[ht]
	\centering
	\includegraphics[width=\linewidth]{partition/img/Ncell.png}  
%	\caption{this is a figure demo}
%	\label{fig:label}
	\end{figure}


\end{columns}


\end{frame}  

%topics
\section{SVM}

\begin{frame}
	
  \centerline{\textbf{\Huge{SVM}}}
           
	\begin{figure}[ht]
	\centering
	\includegraphics[width=0.5\linewidth]{partition/img/svm_0.png}  
%	\caption{this is a figure demo}
%	\label{fig:label}
	\end{figure}

\end{frame} 

\begin{frame} 
\frametitle{支持向量机}  
\begin{columns}

\column{.5\textwidth}
	\begin{itemize}
		\item SVM, 俗称支持向量机,为一种supervised learning算法,属于classification的范畴。
		\item 在数据挖掘的应用中,与unsupervised的Clustering相对应和区别。
		\item 广泛应用于机器学习(Machine Learning), 计算机视觉(Computer Vision) 和数据挖掘(Data Mining)当中。	
	\end{itemize}

\column{.5\textwidth}
	\begin{figure}[ht]
	\centering
	\includegraphics[width=\linewidth]{partition/img/svm_1.jpg}  
%	\caption{this is a figure demo}
%	\label{fig:label}
	\end{figure}


\end{columns}
\end{frame} 

\begin{frame}
\frametitle{一个游戏}	
现在桌子上有两种颜色的球,现在要把他们分开。
           
	\begin{figure}[ht]
	\centering
	\includegraphics[width=0.5\linewidth]{partition/img/svm_2.jpg}  
%	\caption{this is a figure demo}
%	\label{fig:label}
	\end{figure}

\end{frame} 

\begin{frame}
\frametitle{一个游戏}	
我们把一根棍子放在中间,看上去干得不错。
           
	\begin{figure}[ht]
	\centering
	\includegraphics[width=0.5\linewidth]{partition/img/svm_3.jpg}  
%	\caption{this is a figure demo}
%	\label{fig:label}
	\end{figure}

\end{frame} 

\begin{frame}
\frametitle{一个游戏}	
有些人又往桌子上放了一些球,大部分都分对了,但是出现了一个错分的。
           
	\begin{figure}[ht]
	\centering
	\includegraphics[width=0.5\linewidth]{partition/img/svm_4.jpg}  
%	\caption{this is a figure demo}
%	\label{fig:label}
	\end{figure}

\end{frame} 

\begin{frame}
\frametitle{一个游戏}	
SVM就是试图把棍放在最佳位置,好让在棍的两边有尽可能大的间隙。
           
	\begin{figure}[ht]
	\centering
	\includegraphics[width=0.5\linewidth]{partition/img/svm_5.jpg}  
%	\caption{this is a figure demo}
%	\label{fig:label}
	\end{figure}

\end{frame} 

\begin{frame}
\frametitle{一个游戏}	
现在即使放了更多的球,棍仍然是一个好的分界线
           
	\begin{figure}[ht]
	\centering
	\includegraphics[width=0.5\linewidth]{partition/img/svm_6.jpg}  
%	\caption{this is a figure demo}
%	\label{fig:label}
	\end{figure}

\end{frame} 

\begin{frame}
\frametitle{一个游戏}	
 我们已经学会了一个trick,然后,在SVM 工具箱中有另一个更加重要的 trick,于是又有一个新的挑战。
           
	\begin{figure}[ht]
	\centering
	\includegraphics[width=0.5\linewidth]{partition/img/svm_7.jpg}  
%	\caption{this is a figure demo}
%	\label{fig:label}
	\end{figure}

\end{frame} 

\begin{frame}
\frametitle{一个游戏}	
现在,没有棍可以很好地分开两种球了,现在怎么办呢?我们可以一拍桌子,球飞到空中。然后,抓起一张纸,插到了两种球的中间。
           
	\begin{figure}[ht]
	\centering
	\includegraphics[width=0.5\linewidth]{partition/img/svm_8.jpg}  
%	\caption{this is a figure demo}
%	\label{fig:label}
	\end{figure}

\end{frame} 

\begin{frame}
\frametitle{一个游戏}	
现在,这些球看起来像是被一条曲线分开了。
           
	\begin{figure}[ht]
	\centering
	\includegraphics[width=0.5\linewidth]{partition/img/svm_9.jpg}  
%	\caption{this is a figure demo}
%	\label{fig:label}
	\end{figure}

\end{frame} 


\begin{frame}
\frametitle{怎么求解SVM?}
给定训练样本集$D = \{(x_1,y_1),(x_2,y_2),\dots,(x_m,y_m)\}$
 如果一组二分类的样本可以用过一个线性超平面将其完全分开,那么这个划分的超平面可以通过线性方程来表示:
 $$\boldsymbol{w}^T\boldsymbol{x}+b=0$$
 $$w^Tx +b = 0$$

           
	\begin{figure}[ht]
	\centering
	\includegraphics[width=0.5\linewidth]{partition/img/svm_10.jpg}  
%	\caption{this is a figure demo}
%	\label{fig:label}
	\end{figure}

\end{frame} 


\end{document}  